\documentclass[12pt]{article}
\usepackage[colorlinks,breaklinks,linkcolor=red,citecolor=blue]
{hyperref} 
\usepackage{charter}
\def\itemautorefname~#1\null{(#1)\null}
\usepackage{eulervm}
\usepackage[letterpaper,margin=.75in]{geometry}
\title{Sketch of our contract}
\author{Marcus Bishop and Alexis Sparko}
\begin{document}
\maketitle

\section{To be included in the contract}
\begin{enumerate}

\item We agree that Alexis will pay \$1000 and Marcus \$1500 per 
month to the mortgage, to be paid in two installments every month. 
Alexis will submit the payments on the first and the fifteenth of 
every month. Marcus will transfer his share to our joint account on 
the 31st and the 14th of every month.

\item We will both pay \$500 per month to our joint account for house 
expenses. We will contribute more to the joint account as needed, but 
only if mutually agreeable, and in equal proportions.

\item\label{ledger} We agree to keep a detailed ledger recording all 
the money each of us has spent on the house, including closing costs, 
mortgage payments, repairs, and renovations, but not including 
furniture or decoration. In this way we will always be able to calculate
the proportion of the total worth of the home each of us owns 
individually, as described in \autoref{calculation} below.

\item\label{calculation} We agree to calculate our equity in the house using
the following formulae. Let $p_1,p_2,\ldots,p_m$ be the payments
made by Alexis and let $s_i$ be the number of days since payment
$p_i$ for all $1\le i\le m$. Simlarly, let $q_1,q_2,\ldots,q_n$
be the payments made by Marcus and let $t_i$ be the number of days
since payment $q_i$ for $1\le i\le n$. Then we calculate
the equities $e_A,e_M,e_B$ of Alexis, Marcus, and the Bank respectively
using the formulae
\[e_A=\sum_{i=1}^mp_iR^{s_i},\qquad
e_M=\sum_{i=1}^nq_iR^{t_i},\qquad
e_B=371060R^u-e_A-e_M\]
where $u$ is the number of days since 30~May~2016 and
$R=1+\frac{.03625}{365}$.

\item If one of us fails to make a mortgage payments, the other will 
make full payments, until two months have passed, recording the full 
payments in the ledger \autoref{ledger}
as payments made soley by him or her.
However, after two months have passed,
the delinquent owner will record {\em negative} 50\% 
of his or her usual payment in the ledger, while the paying owner
will continue to record his or her full payment.

\item Neither owner will sell his or her share of the house in the 
first five years of ownership, except in cases of medical emergency, 
death of a family member, or loss of job. After five years either 
owner can sell his or her share of the house, but only
to the other owner. The house will be appraised and the 
buyer will pay the seller whichever of the following is the lesser. 

\begin{enumerate} \item The amount the seller has already contributed 
to the house, $e_A$ or $e_M$ from \autoref{calculation}

\item The same proportion 
of the appraisal value $P$ that the seller has already contributed to the 
house, which is $\frac{e_AP}{e_A+e_M+e_B}$ for Alexis or
$\frac{e_MP}{e_A+e_M+e_B}$ for Marcus.

\end{enumerate} 
\end{enumerate}

\section{Not necessarily included in the contract}
\begin{enumerate}

\item Alexis makes all decoration decisions.

\item No out-of-town guest of one owner will stay in the house for 
more than ten consecutive days without the consent of the 
other owner.

\item The total number of nights any owner's out-of-town guests stay 
in the house will not exceed 30~nights per year.

\item Local guests will not spend the night more than two nights per 
week.

\item All house parties will be organized with the consent of both 
owners.

\end{enumerate}
\end{document}
